% Options for packages loaded elsewhere
\PassOptionsToPackage{unicode}{hyperref}
\PassOptionsToPackage{hyphens}{url}
%
\documentclass[
]{article}
\title{Travail de session BIO500 - Une abondance d'oiseaux du Nord au
Sud du Québec}
\author{Frédérick St-Pierre \and Yohan Wegener \and Félix
Labbé \and Aurel Veillet}
\date{2024-04-24}

\usepackage{amsmath,amssymb}
\usepackage{lmodern}
\usepackage{iftex}
\ifPDFTeX
  \usepackage[T1]{fontenc}
  \usepackage[utf8]{inputenc}
  \usepackage{textcomp} % provide euro and other symbols
\else % if luatex or xetex
  \usepackage{unicode-math}
  \defaultfontfeatures{Scale=MatchLowercase}
  \defaultfontfeatures[\rmfamily]{Ligatures=TeX,Scale=1}
\fi
% Use upquote if available, for straight quotes in verbatim environments
\IfFileExists{upquote.sty}{\usepackage{upquote}}{}
\IfFileExists{microtype.sty}{% use microtype if available
  \usepackage[]{microtype}
  \UseMicrotypeSet[protrusion]{basicmath} % disable protrusion for tt fonts
}{}
\makeatletter
\@ifundefined{KOMAClassName}{% if non-KOMA class
  \IfFileExists{parskip.sty}{%
    \usepackage{parskip}
  }{% else
    \setlength{\parindent}{0pt}
    \setlength{\parskip}{6pt plus 2pt minus 1pt}}
}{% if KOMA class
  \KOMAoptions{parskip=half}}
\makeatother
\usepackage{xcolor}
\IfFileExists{xurl.sty}{\usepackage{xurl}}{} % add URL line breaks if available
\IfFileExists{bookmark.sty}{\usepackage{bookmark}}{\usepackage{hyperref}}
\hypersetup{
  pdftitle={Travail de session BIO500 - Une abondance d'oiseaux du Nord au Sud du Québec},
  pdfauthor={Frédérick St-Pierre; Yohan Wegener; Félix Labbé; Aurel Veillet},
  hidelinks,
  pdfcreator={LaTeX via pandoc}}
\urlstyle{same} % disable monospaced font for URLs
\usepackage[margin=1in]{geometry}
\usepackage{graphicx}
\makeatletter
\def\maxwidth{\ifdim\Gin@nat@width>\linewidth\linewidth\else\Gin@nat@width\fi}
\def\maxheight{\ifdim\Gin@nat@height>\textheight\textheight\else\Gin@nat@height\fi}
\makeatother
% Scale images if necessary, so that they will not overflow the page
% margins by default, and it is still possible to overwrite the defaults
% using explicit options in \includegraphics[width, height, ...]{}
\setkeys{Gin}{width=\maxwidth,height=\maxheight,keepaspectratio}
% Set default figure placement to htbp
\makeatletter
\def\fps@figure{htbp}
\makeatother
\setlength{\emergencystretch}{3em} % prevent overfull lines
\providecommand{\tightlist}{%
  \setlength{\itemsep}{0pt}\setlength{\parskip}{0pt}}
\setcounter{secnumdepth}{-\maxdimen} % remove section numbering
\ifLuaTeX
  \usepackage{selnolig}  % disable illegal ligatures
\fi

\begin{document}
\maketitle

\hypertarget{ruxe9sumuxe9}{%
\subsection{Résumé}\label{ruxe9sumuxe9}}

This study investigates the abundance of birds from Northern to Southern
Quebec. We analyze the distribution patterns of various bird species and
assess their diversity across different ecological regions. Our findings
highlight the importance of conservation efforts in preserving avian
biodiversity in Quebec.

\hypertarget{introduction}{%
\subsection{Introduction}\label{introduction}}

Afin de comprendre la dynamique de la biodiversité aviaire du Québec sur
une période donnée, nous avons établi une question générale composée de
deux sous-questions nous permettant d'y parvenir: Y a-t-il une
corrélation entre la richesse spécifique et l'abondance des espèces
d'oiseaux au Québec? Quelle est la tendance de la richesse aviaire
spécifique du Québec de 2016 à 2020? Quelle est la tendance d'abondance
selon la latitude?

La première question cherche à établir s'il existe une corrélation entre
la richesse spécifique (le nombre total d'espèces) et l'abondance (la
quantité relative de chaque espèce). Comprendre cette relation est
crucial pour évaluer la santé globale d'un écosystème et comment il
pourrait être affecté par des changements environnementaux.

La deuxième question cherche à analyser la tendance de la richesse
aviaire spécifique dans la région du Québec au cours des années 2016 à
2020. Cette analyse temporelle permettrait de détecter des changements
survenus dans la diversité des espèces aviaires au fil du temps, ce qui
pourrait être lié à des facteurs tels que le changement climatique, la
dégradation de l'habitat ou d'autres pressions environnementales.

Enfin, la troisième question explore la variabilité saisonnière de la
richesse aviaire spécifique. La diversité des espèces aviaires varie
tout au long de l'année à la suite des changements de saison. Les
résultats peuvent fournir des informations importantes sur les schémas
de migration, les cycles de reproduction et les fluctuations des
populations, ce qui est crucial pour la conservation et la gestion des
écosystèmes. En somme, ces questions cherchent à éclairer les liens
complexes entre la biodiversité aviaire, le temps et l'environnement
dans la région du Québec.

\hypertarget{muxe9thode}{%
\subsection{Méthode}\label{muxe9thode}}

Les données utilisées dans ce travail représentent la composition et la
phénologie sonore des oiseaux au Québec, recueillies dans le cadre d'un
programme de surveillance de la biodiversité acoustique mené par le
ministère de l'Environnement, de la Lutte contre les Changements
Climatiques, de la Faune et des Parcs (MELCCFP), dans le contexte du
Réseau de suivi de la biodiversité du Québec. Elles répertorient les
observations sonores des oiseaux. Les inventaires acoustiques sont
réalisés au moyen d'enregistreurs sonores qui capturent les cris et les
chants des oiseaux. Ces enregistrements représentent des efforts
d'échantillonnage et sont ensuite analysés par un spécialiste en
taxonomie qui identifie les espèces d'oiseaux enregistrées. Lorsque
possible, l'heure de l'observation est également enregistrée
(time\_obs). Ainsi, ces données ont été analysées grâce au logiciel R
afin de créer des figures qui représentent les résultats correspondant
aux trois questions énoncées ci-dessus.

\hypertarget{ruxe9sultats}{%
\subsection{Résultats}\label{ruxe9sultats}}

À la suite de la création d'une base de données et des analyses dans
RStudio, trois figures ont été créées afin de mieux répondre à nos
questions.

\begin{itemize}
\tightlist
\item
  La @ref(Fig1) présente le nombre d'espèces présentes en fonction de la
  latitude du Québec. La régression en bleu permet de voir la moyenne
  d'espèces uniques. Les points présents dans la figure représentent
  chaque station d'écoute.
\end{itemize}

\hypertarget{figure-1}{%
\subsection{Figure 1}\label{figure-1}}

\begin{verbatim}
## `geom_smooth()` using formula = 'y ~ x'
\end{verbatim}

\includegraphics{rapportb_files/figure-latex/Fig1-1.pdf}

\begin{itemize}
\tightlist
\item
  La Figure 2 (@ref(Fig2)) présente l'abondance des observations de
  parulines, elle aussi en fonction de la latitude. Une régression, en
  bleu, avec l'incertitude, en gris, est observable.
\end{itemize}

\hypertarget{figure-2}{%
\subsection{Figure 2}\label{figure-2}}

\includegraphics{rapportb_files/figure-latex/Fig2-1.pdf}

\begin{itemize}
\tightlist
\item
  La Figure 3 (@ref(Fig3))présente le nombre d'espèces uniques observées
  dans tous les sites d'échantillonnages de l'étude pour chaque année,
  de 2016 à 2020.''
\end{itemize}

\hypertarget{figure-3}{%
\subsection{Figure 3}\label{figure-3}}

\includegraphics{rapportb_files/figure-latex/Fig3-1.pdf}

\hypertarget{discussion}{%
\subsection{Discussion}\label{discussion}}

En suivant l'abondance d'une seule famille d'oiseau, comme les
parulines, il est possible de comprendre sa répartition et de confirmer
des ``hot spots''. En se fiant à la figure 2 (@ref()), on observe que
les parulines suivent une tendance négative en fonction de la latitude.
Bien que l'incertitude de ces résultats soit très large, il est certain
que le sud du Québec soit plus abondant en paruline et il est possible
que cette famille disparaisse après le 58ième parallèle. Il est
important de mentionner qu'aucun relevé terrain a été fait entre les
latitudes 51 et 55 comme il est possible de voir à la figure 1. Cette
tendance d'abondance se maintenait pour l'entièreté des observations, ce
qui correspond à la littérature scientifique, puisque le climat tempéré
du sud du Québec favorise l'abondance et la répartition d'oiseaux alors
que le climat davantage froid et rude du nord du Québec aurait tendance
à limiter l'abondance des oiseaux.Cependant cette tendance risque de
changer drastiquement avec l'augmentation des températures au Québec ce
qui favorisera la migration des espèces davantage vers le nord
(Gahbauer, et al., 2022).

La richesse spécifique est toujours importante lorsqu'on parle de
biodiversité. La richesse spécifique aviaire du Québec change beaucoup
en fonction de la latitude. Au sud, soit vers le 45ième parallèle, un
peu plus de 70 espèces ont été observées à une même station
d'échantillonnage (@ref(Fig1)). À l'autre extrême, près du 61ième
parallèle, on y retrouve seulement une dizaine d'espèces aviaires. Les
conditions plus difficiles (froid, absence d'arbre, ressources limités,
etc) s'accentuant en fonction de la latitude créant ce gradient de
richesse spécifique. En comparant la courbe de la figure 1 (@ref(Fig1))
et celle de la figure 2 (@ref(Fig2)), il semble avoir une forte
corrélation entre le nombre d'espèce et l'abondance. C'est deux courbes

Finalement, comme dans de nombreuses autres régions du monde, de
nombreuses espèces d'oiseaux migrateurs et résidents ont connu un déclin
de leurs populations au Québec. Les changements climatiques, la perte
d'habitat, la baisse du succès de reproduction et la hausse de la
prédation des nids sont parmi les facteurs qui peuvent contribuer à ces
déclins bien que ces-derniers soient très variables et difficiles à
mesurer dans le temps. Ainsi, en observant la figure 3 (@ref(FIg3)), on
peut observer une augmentation du nombre d'espèces uniques de 2016
jusqu'en 2019 puis une forte baisse en 2020 de plus de 50 espèces
comparativement à l'année 2019. Cependant, selon la littérature, la
situation semblerait stable en 2019. En effet, la situation est loin
d'être catastrophique car, les tendances stipulent que 50\% des espèces
seraient en hausse et 50\% en déclin (Desrochers, 2019). Cependant,
qu'est ce qui peut expliquer la forte diminution du nombre d'espèces
uniques en 2020? Plusieurs explications sont mises de l'avant telles que
la baisse de recensement d'oiseaux en raison du confinement dû à la
COVID-19. Cependant, d'un point de vue scientifique, une recherche
explique que les oiseaux peuvent vivres un phénomène de périodicité.Par
exemple, il existe des années semencières au Québec où la production de
graines et de cônes est plus importante chez les populations de
conifères et de plantes vivaces (Lacroix-Dubois, 2022). La production
est discontinue car les coûts énergétiques associés à une telle
production dans une année donnée entraînent une diminution de la
production l'année suivante. Par exemple, c'est le cas dans les forêts
boréales du Québec où les conifères présentent généralement des cycles
de 2 à 3 ans. Ces fluctuations de production ont un impact direct sur
les espèces d'oiseaux qui dépendent de ces ressources. En effet, les
années de forte production permettent aux individus de se reproduire sur
une période plus étendue l'année suivante, augmentant ainsi le nombre de
couvées produites. Cependant, on observe ensuite une forte baisse de la
reproduction au cours des années suivantes et cela aurait pu être le cas
pour l'année 2020 (Lacroix-Dubois, 2022).

\hypertarget{bibliographie}{%
\subsection{Bibliographie}\label{bibliographie}}

Desrochers, A., (2019) Les oiseaux du Québec en déclin, vraiment? EBIRD
QUÉBEC.
\url{https://ebird.org/region/CA-QC/post/les-oiseaux-du-quebec-en-declin-vraiment}

Dubé, C., (2023). Protocole d'inventaire acoustique multiespèce avec
appareil Song Meter Mini Bat (SMMB). Ministère de l'Environnement, de la
Lutte contre les changements climatiques, de la Faune et des Parcs
(MELCCFP).
\url{https://mffp.gouv.qc.ca/documents/faune/protocole-inventaire-acoustique-multiespece.pdf}

Gahbauer MA, Parker SR, Wu JX, Harpur C, Bateman BL, et al.~(2022)
Projected changes in bird assemblages due to climate change in a
Canadian system of protected areas. PLOS ONE 17(1): e0262116.
\url{https://doi.org/10.1371/journal.pone.0262116}

Lacroix-Dubois, N. (2022) La périodicité des populations d'oiseaux au
Québec : patrons et causes possibles Mémoire. Maîtrise en sciences
forestières - Université Laval. 37440.pdf (1.22 MB)

\end{document}
